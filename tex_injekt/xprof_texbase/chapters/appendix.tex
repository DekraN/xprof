% !TEX encoding = UTF-8
% !TEX TS-program = pdflatex
% !TEX root = ../InsSchonauerTriadBenchmark.tex
% !TEX spellcheck = it-IT

%************************************************
\chapter{Appendice}
\label{cap:appendix}
%************************************************\\

\section{Codice utilizzato}

\subsection{Profiling}

\subsubsection{Kernel in analisi}

\begin{lstlisting}[language=C,breaklines=true]
#include <stdio.h>
#include <stdlib.h>

#include <sys/time.h>

#if(defined(PERFCOUNT) || defined(PERFCOUNT2))
	#include <papi.h>
	#include <papiStdEventDefs.h>
	#define PERFCOUNTX
#endif

// #define VERBOSE
#define DEFAULT_NMAX 100000
#define DEFAULT_NR DEFAULT_NMAX
#define DEFAULT_INC 10
#define DEFAULT_XIDX 0

#define MAX_PATH_LENGTH 1024

#define ERROR_DESC stderr

// #define WINOS
#define STACKALLOC

#ifdef WINOS 
	#include <windows.h>
#endif

static void dummy(double A[], double B[], double C[], double D[])
{
	return;
}

static double simulation(int N, int R)
{
	int i, j;
	
	#ifdef STACKALLOC
		double A[N];
		double B[N];
		double C[N];
		double D[N];
	#else
		double * A = malloc(N*sizeof(double));
		double * B = malloc(N*sizeof(double));
		double * C = malloc(N*sizeof(double));
		double * D = malloc(N*sizeof(double));
	#endif
	
	double elaps;
	
	for (i = 0; i < N; ++i)
	{
	    A[i] = 0.00;
	    B[i] = 1.00;
	    C[i] = 2.00;
	    D[i] = 3.00;
  	}
	
	#ifdef WINOS
		FILETIME tp;
		GetSystemTimePreciseAsFileTime(&tp);
		elaps = - (double)(((ULONGLONG)tp.dwHighDateTime << 32) | (ULONGLONG)tp.dwLowDateTime)/10000000.0;
	#else
		struct timeval tp;
		gettimeofday(&tp, NULL);
		elaps = -(double)(tp.tv_sec + tp.tv_usec/1000000.0);
	#endif
	
	for(j=0; j<R; ++j)
	{
		for(i=0; i<N; ++i)
			A[i] = B[i] + C[i]*D[i];
			
		if(A[2] < 0) dummy(A, B, C, D);
	}
	
	#ifndef STACKALLOC
		free(A);
		free(B); 
		free(C);
		free(D);
	#endif
	
	#ifdef WINOS
		GetSystemTimePreciseAsFileTime(&tp);
		return elaps + (double)(((ULONGLONG)tp.dwHighDateTime << 32) | (ULONGLONG)tp.dwLowDateTime)/10000000.0;
	#else
		gettimeofday(&tp, NULL);
		return elaps + ((double)(tp.tv_sec + tp.tv_usec/1000000.0));
	#endif
}

enum
{
	NOERROR_EXIT = 0
	#ifdef PERFCOUNTX
	, ERROR_PAPIADDEVENTTOTCYC,
	ERROR_PAPIADDEVENTTOTINS,
	ERROR_PAPIADDEVENTL2TCM,
	ERROR_PAPIADDEVENTL3TCM,
	ERROR_PAPIADDEVENTLDINS,
	ERROR_PAPIADDEVENTSRINS,
	ERROR_PAPIADDEVENTFPOPS,
	ERROR_PAPIADDEVENTFPINS,
	MAX_PAPICOUNTERS = ERROR_PAPIADDEVENTFPINS,
	ERROR_PAPICREATEEVSET,
	ERROR_PAPISTART,
	ERROR_PAPISTOP,
	ERROR_PAPIVER
	#endif
};

#ifdef PERFCOUNTX

#ifdef PERFCOUNT
enum
{
	PAPI_TOTCYC,
	PAPI_TOTINS,
	PAPI_L2TCM,
	PAPI_L3TCM,
	PAPI_LDINS,
	PAPI_SRINS,
	PAPI_FPOPS,
	PAPI_FPINS,
	MAX_PERFCOUNT
};
#else
#ifdef PERFCOUNT2
enum
{
	PAPI_UOPSEXECUTEDSTALLCYCLES,
	PAPI_LLCPREFETCHMISSESPREFETCH,
	PAPI_FPCOMPOPSEXEX87,
	PAPI_FPCOMPOPSEXESSESCALARDOUBLE,
	PAPI_ARITHFPUDIVACTIVE,
	PAPI_ARITHFPUDIV,
	MAX_PERFCOUNT
};
#endif 
#endif
	

static inline void handle_error(int retval)
{
	fprintf(ERROR_DESC, "%s\n", PAPI_strerror(retval));
	return;
}
#endif

int main(int argc, char *argv[])
{
	const int NR = argc > 1 ? atoi(argv[1]) : DEFAULT_NR;
	const int NMAX = argc > 2 ? atoi(argv[2]) : DEFAULT_NMAX;
	const int inc = argc > 3 ? atoi(argv[3]) : DEFAULT_INC;
	
	#ifdef VERBOSE
		const int xidx = argc > 4 ? atoi(argv[4]) : DEFAULT_XIDX;
	#endif
	
	int i, j, k;
	
	#ifdef VERBOSE
		FILE * fp;
		printf("\n*** Schonauer Triad benchmark ***\n");
	
		char csvname[MAX_PATH_LENGTH];
	  	sprintf(csvname, "data%d.csv", xidx);
	
		if(!(fp = fopen(csvname, "a+")))
		{
			printf("\nError whilst writing to file\n");
			return 1;
		}
	#endif
	
	int R, N;
	double MFLOPS;
	double elaps;
	
	#ifdef PERFCOUNTX
		double cpu_time = 0.00;

		printf("\nInitializing PAPI Library..\n");
		printf("PAPI_VER_CURRENT value: %d\n", PAPI_VER_CURRENT);
		int EventSet = PAPI_NULL;
		int retval = PAPI_library_init(PAPI_VER_CURRENT);
		if(retval != PAPI_VER_CURRENT && retval > 0)
		{
			fprintf(ERROR_DESC,"PAPI library version mismatch!\en");
			handle_error(retval);			
			return ERROR_PAPIVER;
		}

		if (retval < 0)
		{
			fprintf(ERROR_DESC, "Some library error occurred: ");
			handle_error(retval);
			return ERROR_PAPIVER;
		}
		retval = PAPI_is_initialized();
		if ((retval = PAPI_is_initialized()) != PAPI_LOW_LEVEL_INITED)
		{
			fprintf(ERROR_DESC, "Some library low-level initialization error occurred: ");
			handle_error(retval);
			return ERROR_PAPIVER;
		}		

		if((retval = PAPI_create_eventset(&EventSet)) != PAPI_OK)
		{
			fprintf(ERROR_DESC, "error on PAPI_create_eventset(): ");
			handle_error(retval);	
			return ERROR_PAPICREATEEVSET;
		}
		if((retval = PAPI_add_event(EventSet, PAPI_TOT_CYC)) != PAPI_OK)
		{
			fprintf(ERROR_DESC, "error on PAPI_add_event(PAPI_TOT_CYC): ");	
			handle_error(retval);				
			return ERROR_PAPIADDEVENTTOTCYC;
		}
		if((retval = PAPI_add_event(EventSet, PAPI_TOT_INS)) != PAPI_OK)
		{
			fprintf(ERROR_DESC, "error on PAPI_add_event(PAPI_TOT_INS): ");
			handle_error(retval);	
			return ERROR_PAPIADDEVENTTOTINS;
		}
		if((retval = PAPI_add_event(EventSet, PAPI_L2_TCM)) != PAPI_OK)
		{
			fprintf(ERROR_DESC, "error on PAPI_add_event(PAPI_L2_TCM): ");
			handle_error(retval);		
			return ERROR_PAPIADDEVENTL2TCM;
		}
		if((retval = PAPI_add_event(EventSet, PAPI_L3_TCM)) != PAPI_OK)
		{
			fprintf(ERROR_DESC, "error on PAPI_add_event(PAPI_L3_TCM): ");
			handle_error(retval);	
			return ERROR_PAPIADDEVENTL3TCM;
		}
		if((retval = PAPI_add_event(EventSet, PAPI_LD_INS)) != PAPI_OK)
		{
			fprintf(ERROR_DESC, "error on PAPI_add_event(PAPI_LD_INS): ");
			handle_error(retval);	
			return ERROR_PAPIADDEVENTLDINS;
		}
		if((retval = PAPI_add_event(EventSet, PAPI_SR_INS)) != PAPI_OK)
		{
			fprintf(ERROR_DESC, "error on PAPI_add_event(PAPI_SR_INS)");
			handle_error(retval);	
			return ERROR_PAPIADDEVENTSRINS;
		}
		if((retval = PAPI_add_event(EventSet, PAPI_FP_OPS)) != PAPI_OK)
		{
			fprintf(ERROR_DESC, "error on PAPI_add_event(PAPI_FP_OPS): ");
			handle_error(retval);	
			return ERROR_PAPIADDEVENTFPOPS;
		}	
		if((retval = PAPI_add_event(EventSet, PAPI_FP_INS)) != PAPI_OK)
		{
			fprintf(ERROR_DESC, "error on PAPI_add_event(PAPI_FP_INS): ");
			handle_error(retval);	
			return ERROR_PAPIADDEVENTFPINS;	
		}
	
		if((retval = PAPI_start(EventSet)) != PAPI_OK)
		{
			fprintf(ERROR_DESC, "error on PAPI_start(): ");
			handle_error(retval);	
			return ERROR_PAPISTART;
		}
	
		// count_init();
		// count_start();
	#endif
	
	for(N=1; N<=NMAX; N += inc)
	{
		R = NR/N;
		elaps = simulation(N, R);
		#ifdef PERFCOUNTX
			cpu_time += elaps;
		#endif
		#ifdef VERBOSE		
			MFLOPS = ((R*N)<<1)/(elaps*1000000);
			fprintf(fp, "%d,%lf\n", N, MFLOPS);
			printf("N = %d, R = %d\n", N, R);
			printf("Elapsed time: %lf\n", elaps);
			printf("MFLOPS: %lf\n", MFLOPS);
		#endif
	}

	#ifdef PERFCOUNTX
		// count_stop();
		// count_finalize();
	
		long_long ret_values[MAX_PERFCOUNT];
		
		if((retval = PAPI_stop(EventSet, ret_values)) != PAPI_OK)
		{
			fprintf(ERROR_DESC, "error on PAPI_stop(): ");
			handle_error(retval);	
			return ERROR_PAPISTOP;
		}


		printf("*** Start PAPI counters listings ***\n");
		#ifdef PERFCOUNT
			printf("PAPI_STOP_OUTPUT: PAPI_TOT_CYC: %lld\n", ret_values[PAPI_TOTCYC]);
			printf("PAPI_STOP_OUTPUT: PAPI_TOT_INS: %lld\n", ret_values[PAPI_TOTINS]);
			printf("PAPI_STOP_OUTPUT: PAPI_L2_TCM: %lld\n", ret_values[PAPI_L2TCM]);
			printf("PAPI_STOP_OUTPUT: PAPI_L3_TCM: %lld\n", ret_values[PAPI_L3TCM]);
			printf("PAPI_STOP_OUTPUT: PAPI_LD_INS: %lld\n", ret_values[PAPI_LDINS]);
			printf("PAPI_STOP_OUTPUT: PAPI_SR_INS: %lld\n", ret_values[PAPI_SRINS]);
			printf("PAPI_STOP_OUTPUT: PAPI_FP_OPS: %lld\n", ret_values[PAPI_FPOPS]);
			printf("PAPI_STOP_OUTPUT: PAPI_FP_INS: %lld\n", ret_values[PAPI_FPINS]);
		#else
		#ifdef PERFCOUNT2
			printf("PAPI_STOP_OUTPUT: UOPS_EXECUTED:STALL-CYCLES: %lld\n", ret_values[PAPI_UOPSEXECUTEDSTALLCYCLES]);
			printf("PAPI_STOP_OUTPUT: LLC-PREFETCH-MISSES: %lld\n", ret_values[PAPI_LLCPREFETCHMISSESPREFETCH]);
			printf("PAPI_STOP_OUTPUT: FP_COMP_OPS_EXE:X87: %lld\n", ret_values[PAPI_FPCOMPOPSEXEX87]);
			printf("PAPI_STOP_OUTPUT: FP_COMP_OPS_EXE:SSE_SCALAR_DOUBLE: %lld\n", ret_values[PAPI_FPCOMPOPSEXESSESCALARDOUBLE]);		
			printf("PAPI_STOP_OUTPUT: ARITH:FPU_DIV_ACTIVE: %lld\n", ret_values[PAPI_ARITHFPUDIVACTIVE]);
			printf("PAPI_STOP_OUTPUT: ARITH:FPU_DIV: %lld\n", ret_values[PAPI_ARITHFPUDIV]);
		#endif
		#endif
		
		printf("\nExecution time: %lf\n\n", cpu_time);
	#endif
	
	#ifdef VERBOSE
		fclose(fp);
		(void) getchar();
	#endif
	return 0;
}

\end{lstlisting}

\subsection{Codice Roofline Model Plotter}

\begin{lstlisting}[language=python,breaklines=true]
#######################################
####### Roofline Model Plotter ########
##### Marco Chiarelli @ UNISALENTO#####
## marco_chiarelli@yahoo.it 23/05/2017#
#######################################
#######################################

import numpy as np

import matplotlib.pyplot as plt
import matplotlib.ticker as mtick

from sys import argv

argc = len(argv)
optimized = argc >= 11

# Verbose Settings
# useful for debugging
verbose_machine = argc > 12 and bool(argv[12]) or True
verbose_kernel = argc > 13 and bool(argv[13]) or True
autosave = argc > 14 and bool(argv[14]) or True

if argc < 9:
    print("Error: Invalid number of arguments.")
    exit(0)
    
# Kernel Info
your_kernel_name = str(argv[1])
your_machine_name = str(argv[5])
Flops = int(argv[2])
Gflops = Flops/(10**9)
L3CM = int(argv[3])
etime = float(argv[4])
GFLOPS = Gflops/etime # y coordinate

global Flops_opt
global Gflops_opt
global L3CM_opt
global etime_opt
global GFLOPS_opt
global traffic_opt
global arithmetic_int_opt

# Machine Info
L3LineSize = int(argv[6]) # LL cache line size
bmax = float(argv[7]) # maximum bandwidth;
peak_fp_performance = float(argv[8]) # GFLOPS/s
ridge_point = peak_fp_performance/bmax
sample_y = round(2*bmax, 2)

if optimized:
    Flops_opt = int(argv[9])
    Gflops_opt = Flops_opt/(10**9)
    L3CM_opt = int(argv[10])
    etime_opt = float(argv[11])
    GFLOPS_opt = Gflops_opt/etime_opt
    traffic_opt = (L3CM_opt*L3LineSize)/(1024**3)
    arithmetic_int_opt = Gflops_opt/traffic_opt # x coordinate

# Some other stuffs
traffic = (L3CM*L3LineSize)/(1024**3)
arithmetic_int = Gflops/traffic

def printline():
    print("--------------------------------------------------")

def printKernel(your_kernel_name, Flops, L3CM, etime, Gflops, GFLOPS, traffic, arithmetic_int):
    printline()
    print(your_kernel_name + " - Roofline Model (RLM)")
    printline()
    print("Floating Point Operations (Flops) = " + str(Flops) + ";")
    print("L3 Cache Misses = " + str(L3CM) + ";")
    print("Execution time = " + str(etime) + ";")
    print("Gflops = " + str(Gflops) + ";")
    print("GFLOPS = " + str(GFLOPS) + ";")
    print("Traffic = " + str(traffic) + " GB;")
    print("Arithmetic Intensity (AI) = " + str(arithmetic_int) + " Gflops/GB.")
    printline()
    print("Your kernel: " + your_kernel_name+";")
    print("is: " + (arithmetic_int > ridge_point and "COMPUTE" or "MEMORY") + " BOUNDED!")
    print("with respect to the machine: ")
    print(your_machine_name+".")
    printline()
    print(" ")
    print(" ")

printline()
print("####### Roofline Model Plotter ########")
print("#######################################")
print("##### Marco Chiarelli @ UNISALENTO#####")
print("## marco_chiarelli@yahoo.it 23/05/2017#")
print("#######################################")
printline()


if verbose_machine == True:
    printline()
    print(your_machine_name)
    print("Listing some machine informations...")
    printline()
    print("L3LineSize = " + str(L3LineSize) + " byte;")
    print("Peak Memory BW = " + str(bmax) + " GB/s;")
    print("Peak FP Performance = " + str(peak_fp_performance) + " Gflops/s;")
    print("Corresponding Ridge point = " + str(ridge_point) + ".")
    printline()
    print(" ")
    print(" ")

if verbose_kernel == True:
    printKernel(your_kernel_name, Flops, L3CM, etime, Gflops, GFLOPS, traffic, arithmetic_int) 

    if optimized == True:
        printKernel("Optimized "+your_kernel_name, Flops_opt, L3CM_opt, etime_opt, Gflops_opt, GFLOPS_opt, traffic_opt, arithmetic_int_opt) 

if optimized == True:
    gain_performance = ((GFLOPS_opt-GFLOPS)/GFLOPS)*100;
    gain_memory = ((arithmetic_int_opt-arithmetic_int)/arithmetic_int)*100;
    print("Percentual Performance Gain: " + str(gain_performance)+ ";")
    print("Percentual Memory Gain: " + str(gain_memory) + ";")
    print("Done a " + (gain_performance > 0 and "good" or "bad") + " job wrt FP performance;")
    print("Done a " + (gain_memory > 0 and "good" or "bad") + " job wrt memory.")
    print(" ")
    print(" ")

# Preparing a straight line fitting
xP = [ridge_point, 2]
yP = [peak_fp_performance, sample_y]

coefficients = np.polyfit(xP, yP, 1)
polynomial = np.poly1d(coefficients)
x_axis = np.linspace(0,20,100)
peak_memory_BW_plt = polynomial(x_axis)
peak_fp_performance_plt = np.ones(len(x_axis))*peak_fp_performance

# PLOT SETTINGS
logbase = 2
fig, ax = plt.subplots()
# Machine RLM plot
ax.loglog(np.where(x_axis >= ridge_point, 0, x_axis),np.where(peak_memory_BW_plt >= peak_fp_performance, 0, peak_memory_BW_plt),'-',basex=logbase,basey=logbase,linewidth=2.5)
ax.loglog(np.where(x_axis >= ridge_point, x_axis, 0),np.where(peak_fp_performance_plt > peak_memory_BW_plt, 0, peak_fp_performance_plt),'-',basex=logbase,basey=logbase,linewidth=2.5)
ax.loglog(ridge_point, peak_fp_performance,'+',basex=logbase,basey=logbase,markeredgewidth=3,markersize=10)
# Kernel(s) plot
ax.loglog(arithmetic_int, GFLOPS,'o',basex=logbase,basey=logbase)
if optimized == True:
    ax.loglog(arithmetic_int_opt, GFLOPS_opt,'o',basex=logbase,basey=logbase)

def ticks(y, pos):
    return r'$2^{:.0f}$'.format(np.log2(y))

#ax.xaxis.set_major_formatter(mtick.FuncFormatter(ticks))
#ax.yaxis.set_major_formatter(mtick.FuncFormatter(ticks))

# PLOT
plt.grid(True, which="both", ls="-")
plt.xlabel("Arithmetic Intensity AI [GFlops/GB]")
plt.ylabel("Attainable Performance [GFLOPS = GFlops/s]")
plt.legend(optimized == True and ["Peak Memory BW", "Peak FP Performance", "Ridge Point", "Your Kernel", "Your Optimized Kernel"] or ["Peak Memory BW", "Peak FP Performance", "Ridge Point", "Your Kernel"], loc='best')
plt.title("Roofline Model (RLM) of: " + your_kernel_name)

if autosave:
    fig.savefig("rlm.png", dpi=96 * 10)
else:
    plt.show()

print("Thank you for using this program.")
print(" ")
print(" ")

\end{lstlisting}

\subsection{Xprof}

\subsubsection{gathering.sh}
\begin{lstlisting}[language=bash,breaklines=true]
#!/bin/bash

#########################################
#### XPROF - Automatic Profiling Tool####
##### Final Built, v0.1 - 12/10/2017 ####
########## Authors/Developer: ###########
### Marco Chiarelli    @ UNISALENTO&CMCC#
### Paolo Panarese     @ UNISALENTO&CMCC#
### Emanuele C. Cesari @ UNISALENTO&CMCC#
######  marco_chiarelli@yahoo.it   ######
######  panarese.paolo@gmail.com   ######
######     cce.9123@gmail.com      ######
#########################################
#########################################

#######################################
####### PERFORMANCE GATHERING #########
###### Paolo Panarese @ UNISALENTO#####
## panarese.paolo@gmail.com 23/05/2017#
#######################################
#######################################

function checkStatus {
	if [[ $1 -gt 0 ]]; then
		echo "Error reached, Quit!"
		exit $1
	fi
}

if [[ $# -lt 1 ]]; then
	echo "INSERT SOURCE FILE"
	exit 1
fi

echo
echo
echo "#########################################"
echo "#### XPROF - Automatic Profiling Tool####"
echo "##### Final Built, v0.1 - 12/10/2017 ####"
echo "########## Authors/Developer: ###########"
echo "### Marco Chiarelli    @ UNISALENTO&CMCC#"
echo "### Paolo Panarese     @ UNISALENTO&CMCC#"
echo "### Emanuele C. Cesari @ UNISALENTO&CMCC#"
echo "######  marco_chiarelli@yahoo.it   ######"
echo "######  panarese.paolo@gmail.com   ######"
echo "######     cce.9123@gmail.com      ######"
echo "#########################################"
echo "#########################################"

echo
echo
echo "#######################################"
echo "####### PERFORMANCE GATHERING #########"
echo "###### Paolo Panarese @ UNISALENTO#####"
echo "## panarese.paolo@gmail.com 23/05/2017#"
echo "#######################################"
echo "#######################################"
echo
echo

NAME=$1
FILENAME="$NAME.c"
PAPI_DIR=$(which papi_avail | sed -e 's/\/bin\/papi_avail/\//g')

if [ -f $FILENAME ]; then
   echo "File $FILENAME exists."
else
   echo "File $FILENAME does not exist."
fi
OPT=""
if [[ $# -gt 1 ]]; then
	OPT=${@:2}
fi
echo "$FILENAME $OPT"

ulimit -s unlimited

echo "Compiling with gprof-oprof flags"
### Compile
gcc -pg -g -O0 $FILENAME -lm -o $NAME $OPT
checkStatus $?

# GPROF
## Dependencies:
# apt install graphviz xdot
# pip install graphviz xdot gprof2dot

echo
echo "********* gprof *********"
### Execute compiled file
./$NAME
checkStatus $?

echo "Creating gprof dir"
rm -r -f gprof
mkdir -p gprof
echo "Running gprof"
gprof ./$NAME > gprof/gprof.out
mv gmon.out gprof/gmon.out
rm -f gmon.out

# OPROF

echo
echo "********* oprof *********"
operf ./$NAME
checkStatus $?
opreport
checkStatus $?
opannotate --source --output-dir=output_oprof `./$NAME`
checkStatus $?

echo "Creating oprof dir and correctly placing oprof.out" 
rm -r -f oprof
mkdir -p oprof
cp output_oprof`pwd`/$FILENAME oprof/oprof.out
echo "Removing some useless stuffs."
rm -r oprofile_data
rm -r output_oprof


# PERFCOUNT

echo
echo "********* perfcount *********"
echo "Creating papi dir."
rm -r -f papi
mkdir -p papi

echo "Compiling base kernel with PAPI (advanced) flags"
echo "gcc -I$PAPI_DIR""include -L$PAPI_DIR""lib -O0 -lm -o $NAME $FILENAME -lpapi -D PERFCOUNT2"
gcc -I$PAPI_DIR""include -L$PAPI_DIR""lib -O0 -lm -o $NAME $FILENAME -lpapi -D PERFCOUNT2
checkStatus $?

echo "Compiling optimized kernel with PAPI (advanced) flags"
echo "gcc -I$PAPI_DIR""include -L$PAPI_DIR""lib -O3 -march=native -lm -o "$NAME"_opt $FILENAME -lpapi -D PERFCOUNT2"
gcc -I$PAPI_DIR""include -L$PAPI_DIR""lib -O3 -march=native -lm -o "$NAME"_opt $FILENAME -lpapi -D PERFCOUNT2
checkStatus $?

echo "Running base kernel with PAPI (advanced) flags."
echo
./$NAME > papi/perf.out
checkStatus $?

echo "Compiling base kernel with PAPI (basic) flags"
echo "gcc -I$PAPI_DIR""include -L$PAPI_DIR""lib -O0 -lm -o $NAME $FILENAME -lpapi -D PERFCOUNT"
gcc -I$PAPI_DIR""include -L$PAPI_DIR""lib -O0 -lm -o $NAME $FILENAME -lpapi -D PERFCOUNT
checkStatus $?

./$NAME >> papi/perf.out
checkStatus $?

echo "Running optimized kernel with PAPI (advanced) flags."
echo

./"$NAME"_opt > papi/perf_opt.out
checkStatus $?

echo "Compiling optimized kernel with PAPI (basic) flags"
echo "gcc -I$PAPI_DIR""include -L$PAPI_DIR""lib -O3 -march=native -lm -o "$NAME"_opt $FILENAME -lpapi -D PERFCOUNT"
gcc -I$PAPI_DIR""include -L$PAPI_DIR""lib -O3 -march=native -lm -o "$NAME"_opt $FILENAME -lpapi -D PERFCOUNT
checkStatus $?

./"$NAME"_opt >> papi/perf_opt.out
checkStatus $?

HOSTNAME=`hostname`
CACHE_LINE=`cat /proc/cpuinfo | grep cache_alignment | sed 's/^.*:\ //' | sort -u`
B_MAX=3.82828
PEAK_FP=20.8

cd papi

echo "hostname: "$HOSTNAME > machine.out
echo "kernel_name: "$NAME >> machine.out
echo "cache_line: "$CACHE_LINE >> machine.out
echo "b_max: "$B_MAX >> machine.out
echo "peak_fp: "$PEAK_FP >> machine.out

cd ..

echo "Successfully Ended."


\end{lstlisting}

\subsubsection{analyzer.sh}
\begin{lstlisting}[language=bash,breaklines=true]
#!/bin/bash

#########################################
#### XPROF - Automatic Profiling Tool####
##### Final Built, v0.1 - 12/10/2017 ####
########## Authors/Developer: ###########
### Marco Chiarelli    @ UNISALENTO&CMCC#
### Paolo Panarese     @ UNISALENTO&CMCC#
### Emanuele C. Cesari @ UNISALENTO&CMCC#
######  marco_chiarelli@yahoo.it   ######
######  panarese.paolo@gmail.com   ######
######     cce.9123@gmail.com      ######
#########################################
#########################################

#######################################
####### PERFORMANCEANALYZER ###########
###### Paolo Panarese @ UNISALENTO#####
## panarese.paolo@gmail.com 23/05/2017#
#######################################
#######################################

function checkStatus {
	if [[ $1 -gt 0 ]]; then
		echo "Error reached, Quit!"
		exit $1
	fi
}

echo
echo
echo "#########################################"
echo "#### XPROF - Automatic Profiling Tool####"
echo "##### Final Built, v0.1 - 12/10/2017 ####"
echo "########## Authors/Developer: ###########"
echo "### Marco Chiarelli    @ UNISALENTO&CMCC#"
echo "### Paolo Panarese     @ UNISALENTO&CMCC#"
echo "### Emanuele C. Cesari @ UNISALENTO&CMCC#"
echo "######  marco_chiarelli@yahoo.it   ######"
echo "######  panarese.paolo@gmail.com   ######"
echo "######     cce.9123@gmail.com      ######"
echo "#########################################"
echo "#########################################"

echo
echo
echo "#######################################"
echo "####### PERFORMANCEANALYZER ###########"
echo "###### Paolo Panarese @ UNISALENTO#####"
echo "## panarese.paolo@gmail.com 23/05/2017#"
echo "#######################################"
echo "#######################################"
echo
echo

### GPROF
## Dependencies:
# apt install graphviz xdot
# pip install graphviz xdot gprof2dot
echo "Generating call graph through gprof2dot and xdot."
cd gprof
### Export to dot file
gprof2dot gprof.out > call_graph.dot
### Convert to png image
dot -Tpng call_graph.dot -o call_graph.png
cd ..

### OPROF
## Dependencies:
# pip install matplotlib
# apt-get install sed
echo "Generating oprof (cumulative) histogram through oprof_graph.py"
cd oprof
cat --number oprof.out | sed 's/\t/\ \ \ \ /' | sed 's/^\ *\([0-9]*\)\ *\([0-9]*\)\ *\([0-9]*\)\ */\1 \2 \3/' | grep ^[0-9]*\ [0-9]*\ [0-9.]*\ *: | sed 's/:.*//' > oprof_data

### print graph with python
python ../oprof_graph.py
checkStatus $?
cd ..


### PERF_RLM
## Dependencies:
# apt install python3-tk
echo "Preparing and exporting vars for TeX Injekt."
cd papi
export GET_DEVELOPER1="Dott. Marco Chiarelli"
export GET_DEVELOPER2="Dott. Paolo Panarese"
export GET_DEVELOPER3="Dott. Emanuele Costa Cesari"
export HOSTNAME=`cat machine.out | grep "hostname" | sed 's/^.*:\ //'`
export KERNEL_NAME=`cat machine.out | grep "kernel_name" | sed 's/^.*:\ //'`

export FP_OPS=`cat perf.out | grep FP_OPS | sed 's/^.*:\ //'`
export L3_TCM=`cat perf.out | grep PAPI_L3_TCM | sed 's/^.*:\ //'`
export L2_TCM=`cat perf.out | grep PAPI_L2_TCM | sed 's/^.*:\ //'`
export TOT_CYC=`cat perf.out | grep PAPI_TOT_CYC | sed 's/^.*:\ //'`
export TOT_INS=`cat perf.out | grep PAPI_TOT_INS | sed 's/^.*:\ //'`
export LLC_PM=`cat perf.out | grep LLC-PREFETCH-MISSES | sed 's/^.*:\ //'`
export LD_INS=`cat perf.out | grep PAPI_LD_INS | sed 's/^.*:\ //'`
export SR_INS=`cat perf.out | grep PAPI_SR_INS | sed 's/^.*:\ //'`
export SC=`cat perf.out | grep UOPS_EXECUTED | sed 's/^.*:\ //'`
export EXEC_TIME=`cat perf.out | grep Execution\ time | sed 's/^.*:\ //' | awk '{ sum += $1; n++ } END { if (n > 0) print sum / n; }'`

export FP_OPS_OPT=`cat perf_opt.out | grep FP_OPS | sed 's/^.*:\ //'`
if [[ $FP_OPS_OPT -le $FP_OPS ]]; then	
	FP_OPS_OPT=$FP_OPS
fi
export L3_TCM_OPT=`cat perf_opt.out | grep PAPI_L3_TCM | sed 's/^.*:\ //'`
export L2_TCM_OPT=`cat perf_opt.out | grep PAPI_L2_TCM | sed 's/^.*:\ //'`
export TOT_CYC_OPT=`cat perf_opt.out | grep PAPI_TOT_CYC | sed 's/^.*:\ //'`
export TOT_INS_OPT=`cat perf_opt.out | grep PAPI_TOT_INS | sed 's/^.*:\ //'`
export LLC_PM_OPT=`cat perf_opt.out | grep LLC-PREFETCH-MISSES | sed 's/^.*:\ //'`
export LD_INS_OPT=`cat perf_opt.out | grep PAPI_LD_INS | sed 's/^.*:\ //'`
export SR_INS_OPT=`cat perf_opt.out | grep PAPI_SR_INS | sed 's/^.*:\ //'`
export SC_OPT=`cat perf_opt.out | grep UOPS_EXECUTED | sed 's/^.*:\ //'`
export EXEC_TIME_OPT=`cat perf_opt.out | grep Execution\ time | sed 's/^.*:\ //' | awk '{ sum += $1; n++ } END { if (n > 0) print sum / n; }'`

export CACHE_LINE=`cat machine.out | grep cache_line | sed 's/^.*:\ //'`
export B_MAX=`cat machine.out | grep b_max | sed 's/^.*:\ //'`
export PEAK_FP=`cat machine.out | grep peak_fp | sed 's/^.*:\ //'`

echo "Generating Roofline Model Plot through Python RLMP".
python3 ../rlmp.py $KERNEL_NAME $FP_OPS $L3_TCM $EXEC_TIME $HOSTNAME $CACHE_LINE $B_MAX $PEAK_FP $FP_OPS_OPT $L3_TCM_OPT $EXEC_TIME_OPT > rlm.out

checkStatus $?

cd ../tex_injekt

echo "Running TeX Injekt."
./tex_injekt.sh

cd ..

echo "Successfully ended."
echo
echo "Thanks you for using this program."
echo
echo

\end{lstlisting}

\subsubsection{OProf Graph}
\begin{lstlisting}[language=python,breaklines=true]
import numpy as np
import matplotlib.pyplot as plt

graphXDelay = 3

x = []
y = []
xLabel = []

file = open("oprof_data", "r") 
lines = file.readlines()

for line in lines:
	splitted = str(line).split(' ')
	x.append(int(splitted[0]))
	if len(x) % graphXDelay == 0:
		xLabel.append(splitted[0])
	else:
		xLabel.append("")
	if len(splitted) > 3:
		y.append(float(splitted[1]))
	else:
		y.append(0)

ind = np.arange(len(y))

width = 0.35       # the width of the bars

fig, ax = plt.subplots()

rects = ax.bar(ind, y, width, color='b')

lineY = np.cumsum(y)
lineGraph = ax.plot(ind, lineY, 'r')

ax.set_title('oprof')
ax.set_xticks(ind + width / 2)
ax.set_xticklabels(xLabel)

ax.legend((rects[0], lineGraph[0]), ("Samples", "Cumulative Sum"))

#plt.show()
#fig.savefig('oprof.png')
fig.savefig('oprof.png', dpi=96 * 10)




\end{lstlisting}

\section{Strumenti utilizzati}

Si ringraziano i rispettivi autori dei seguenti strumenti, soprattutto quelli open-source. 

\begin{itemize}
\item\emph{Dev-C++ 5.7.1};
\item\emph{Windows 10 Pro 64-Bit};
\item\emph{CentOS release 6.5 (Final)};
\item\emph{Cluster Manager v7.0};
\item\emph{CentOS Linux 7 (Core) (7.3.1611, Core)};
\item\emph{CentOS Linux 7 (Core) (7.2.1511, Core)};
\item\emph{Ubuntu Linux 17.04 Zesty Zapus};
\item\emph{OpenVPN GUI v11.6.0.0};
\item\emph{PuTTY Release 0.68};
\item\emph{GCC (tdm64-2) 4.8.1: TDM-GCC (MinGW, Windows, Lenovo Z500)};
\item\emph{gcc (GCC) 4.8.2: Linux (Macchina E4)};
\item\emph{gcc (GCC) 4.8.5 20150623 (Red Hat 4.8.5-11) (iDataPlex dx360M4, nodo Athena)};
\item\emph{gcc (GCC) 4.8.5 20150623 (Red Hat 4.8.5-11) (HP DL560 gen8, server HP)};
\item\emph{gcc (Ubuntu 6.3.0-12ubuntu2) 6.3.0 20170406 (Asus N550 e Lenovo Z500)};
\item\emph{Texmaker 4.5};
\item\emph{benchmark\_visualizer.py};
\item\emph{GNU gprof version 2.25.1-22.base.e17};
\item\emph{operf: oprofile 1.1.0};
\item\emph{Libreria PAPI};
\item\emph{Microsoft Excel 2007 (12.0.6759.5000) SP3 MSO (12.0.6759.5000)};
\item\emph{LibreOffice Calc 5.3.1.2};
\item\emph{Python 3.4.3};
\item\emph{Python IDLE 3.4.3};
\item\emph{Roofline Model Plotter (rlmp.py)};
\item\emph{xprof 0.1a};
\end{itemize}
